\counterwithin{figure}{section}
\section{Problem Definition, Objectives and Bonus Points}

In the project, we are to do a forward converter design with specific properties.


\begin{table}[H]
\centering
\caption{Parameters of the project}
\label{tab:project_spec}
\resizebox{10 cm  }{!}{%
\begin{tabular}{|c|c|}
\hline
Minimum Input Voltage (V)             & \textbf{24} \\ \hline
Maximum Input Voltage (V)             & \textbf{48} \\ \hline
Output Voltage (V)                    & \textbf{15} \\ \hline
Output Power (W)                      & \textbf{48} \\ \hline
Output Volt. Peak-to-Peak Ripple (\%) & \textbf{2}  \\ \hline
Line Regulation (\%)                  & \textbf{2}  \\ \hline
Load Regulation (\%)                  & \textbf{2}  \\ \hline

\end{tabular}}
\end{table}

Some of the points and objectives can be itemized as:

\begin{itemize}
    \item Closed loop is a must and we are not allowed to use digital controllers
    \item PCB design of the converter will be presented.
    \item Bill of material will be presented
    \item Closed loop compansator design with bode-plot is a \textbf{bonus}
    \item We are to do thermal design and efficiency analysis of the project
\end{itemize}

We have mentioned couple of specifications of the project. Our goal was to accomplish the main specifications and add as much as bonus we can. 

In the next section we are introduce a forward converter.
